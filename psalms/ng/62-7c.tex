2. Sitívit in te \textbf{á}nima \textbf{me}a,~*  quam multiplíciter tibi \textbf{ca}ro \textbf{me}a.\

3. In terra desérta, et ínvia, et inaquósa: \dag\  sic in sancto ap\textbf{pá}rui \textbf{ti}bi,~*  ut vidérem virtútem tuam, et \textbf{gló}riam \textbf{tu}am.\

4. Quóniam mélior est misericórdia tua \textbf{su}per \textbf{vi}tas:~*  lábia \textbf{me}a lau\textbf{dá}bunt te.\

5. Sic benedícam te in \textbf{vi}ta \textbf{me}a:~*  et in nómine tuo levábo \textbf{ma}nus \textbf{me}as.\

6. Sicut ádipe et pinguédine repleátur \textbf{á}nima \textbf{me}a:~*  et lábiis exsultatiónis lau\textbf{dá}bit os \textbf{me}um.\

7. Si memor fui tui super stratum meum, \dag\  in matutínis medi\textbf{tá}bor \textbf{in} te:~*  quia fuísti ad\textbf{jú}tor \textbf{me}us.\

8. Et in velaménto alárum tuárum exsultábo, \dag\  adhǽsit ánima \textbf{me}a \textbf{post} te:~*  me suscépit \textbf{déx}tera \textbf{tu}a.\

9. Ipsi vero in vanum quæsiérunt ánimam meam, \dag\  introíbunt in inferi\textbf{ó}ra \textbf{ter}ræ:~*  tradéntur in manus gládii, partes \textbf{vúl}pium \textbf{e}runt.\

10. Rex vero lætábitur in Deo, \dag\  laudabúntur omnes qui \textbf{ju}rant in \textbf{e}o:~*  quia obstrúctum est os loquénti\textbf{um} in\textbf{í}qua.\

