2. Sitívit in te á\textit{ni}\textit{ma} \textbf{me}a,~*  quam multiplíciter tibi caro \textbf{me}a.\

3. In terra desérta, et ínvia, et inaquósa: \dag\  sic in sancto appá\textit{ru}\textit{i} \textbf{ti}bi,~*  ut vidérem virtútem tuam, et glóriam \textbf{tu}am.\

4. Quóniam mélior est misericórdia tua \textit{su}\textit{per} \textbf{vi}tas:~*  lábia mea lau\textbf{dá}bunt te.\

5. Sic benedícam te in \textit{vi}\textit{ta} \textbf{me}a:~*  et in nómine tuo levábo manus \textbf{me}as.\

6. Sicut ádipe et pinguédine repleátur á\textit{ni}\textit{ma} \textbf{me}a:~*  et lábiis exsultatiónis laudábit os \textbf{me}um.\

7. Si memor fui tui super stratum meum, \dag\  in matutínis medi\textit{tá}\textit{bor} \textbf{in} te:~*  quia fuísti adjútor \textbf{me}us.\

8. Et in velaménto alárum tuárum exsultábo, \dag\  adhǽsit ánima \textit{me}\textit{a} \textbf{post} te:~*  me suscépit déxtera \textbf{tu}a.\

9. Ipsi vero in vanum quæsiérunt ánimam meam, \dag\  introíbunt in inferi\textit{ó}\textit{ra} \textbf{ter}ræ:~*  tradéntur in manus gládii, partes vúlpium \textbf{e}runt.\

10. Rex vero lætábitur in Deo, \dag\  laudabúntur omnes qui ju\textit{rant} \textit{in} \textbf{e}o:~*  quia obstrúctum est os loquéntium in\textbf{í}qua.\

