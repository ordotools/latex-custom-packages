2. Vana locúti sunt unusquísque ad \textbf{pró}ximum \textbf{su}um:~*  lábia dolósa, in corde et \textit{cor}\textit{de} \textit{lo}\textbf{cú}ti sunt.\

3. Dispérdat Dóminus univérsa lábi\textbf{a} do\textbf{ló}sa,~*  et \textit{lin}\textit{guam} \textit{ma}\textbf{gní}loquam.\

4. Qui dixérunt: Linguam nostram magnificábimus, \dag\  lábia \textbf{nos}tra a \textbf{no}\textbf{bis} sunt,~*  quis nos\textit{ter} \textit{Dó}\textit{mi}\textbf{nus} est?\

5. Propter misériam ínopum, et \textbf{gé}mitum \textbf{páu}\textbf{pe}rum,~*  nunc exsúr\textit{gam}, \textit{di}\textit{cit} \textbf{Dó}minus.\

6. Ponam in \textbf{sa}lu\textbf{tá}ri:~*  fiduciáliter \textit{a}\textit{gam} \textit{in} \textbf{e}o.\

7. Elóquia Dómini, e\textbf{ló}quia \textbf{cas}ta:~*  argéntum igne examinátum, probátum terræ \textit{pur}\textit{gá}\textit{tum} \textbf{sép}tuplum.\

8. Tu, Dómine, servábis nos: et cus\textbf{tó}di\textbf{es} nos~*  a generatióne \textit{hac} \textit{in} \textit{æ}\textbf{tér}num.\

9. In circúitu \textbf{ím}pii \textbf{ám}\textbf{bu}lant:~*  secúndum altitúdinem tuam multiplicásti \textit{fí}\textit{li}\textit{os} \textbf{hó}minum.\

