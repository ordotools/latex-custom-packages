2. Sitívit in te ánima \textbf{me}a,~*  quam multiplíciter tibi ca\textit{ro} \textbf{me}a.\

3. In terra desérta, et ínvia, et inaquósa: \dag\  sic in sancto appárui \textbf{ti}bi,~*  ut vidérem virtútem tuam, et glóri\textit{am} \textbf{tu}am.\

4. Quóniam mélior est misericórdia tua super \textbf{vi}tas:~*  lábia mea \textit{lau}\textbf{dá}bunt te.\

5. Sic benedícam te in vita \textbf{me}a:~*  et in nómine tuo levábo ma\textit{nus} \textbf{me}as.\

6. Sicut ádipe et pinguédine repleátur ánima \textbf{me}a:~*  et lábiis exsultatiónis laudábit \textit{os} \textbf{me}um.\

7. Si memor fui tui super stratum meum, \dag\  in matutínis meditábor \textbf{in} te:~*  quia fuísti adjú\textit{tor} \textbf{me}us.\

8. Et in velaménto alárum tuárum exsultábo, \dag\  adhǽsit ánima mea \textbf{post} te:~*  me suscépit déxte\textit{ra} \textbf{tu}a.\

9. Ipsi vero in vanum quæsiérunt ánimam meam, \dag\  introíbunt in inferióra \textbf{ter}ræ:~*  tradéntur in manus gládii, partes vúlpi\textit{um} \textbf{e}runt.\

10. Rex vero lætábitur in Deo, \dag\  laudabúntur omnes qui jurant in \textbf{e}o:~*  quia obstrúctum est os loquéntium \textit{in}\textbf{í}qua.\

