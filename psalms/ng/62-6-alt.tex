2. Sitívit in te áni\textit{ma} \textbf{me}a, \ast\  quam multiplíciter tibi \textit{ca}\textit{ro} \textbf{me}a.\

3. In terra desérta, et ínvia, et inaquósa: \dag\  sic in sancto appáru\textit{i} \textbf{ti}bi, \ast\  ut vidérem virtútem tuam, et gló\textit{ri}\textit{am} \textbf{tu}am.\

4. Quóniam mélior est misericórdia tua su\textit{per} \textbf{vi}tas: \ast\  lábia me\textit{a} \textit{lau}\textbf{dá}bunt te.\

5. Sic benedícam te in vi\textit{ta} \textbf{me}a: \ast\  et in nómine tuo levábo \textit{ma}\textit{nus} \textbf{me}as.\

6. Sicut ádipe et pinguédine repleátur áni\textit{ma} \textbf{me}a: \ast\  et lábiis exsultatiónis laudá\textit{bit} \textit{os} \textbf{me}um.\

7. Si memor fui tui super stratum meum, \dag\  in matutínis meditá\textit{bor} \textbf{in} te: \ast\  quia fuísti ad\textit{jú}\textit{tor} \textbf{me}us.\

8. Et in velaménto alárum tuárum exsultábo, \dag\  adhǽsit ánima me\textit{a} \textbf{post} te: \ast\  me suscépit déx\textit{te}\textit{ra} \textbf{tu}a.\

9. Ipsi vero in vanum quæsiérunt ánimam meam, \dag\  introíbunt in inferió\textit{ra} \textbf{ter}ræ: \ast\  tradéntur in manus gládii, partes vúl\textit{pi}\textit{um} \textbf{e}runt.\

10. Rex vero lætábitur in Deo, \dag\  laudabúntur omnes qui jurant \textit{in} \textbf{e}o: \ast\  quia obstrúctum est os loquénti\textit{um} \textit{in}\textbf{í}qua.\

