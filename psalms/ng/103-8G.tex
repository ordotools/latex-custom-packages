2. Confessiónem, et decórem indu\textbf{ís}ti:~*  amíctus lúmine sicut \textit{ves}\textit{ti}\textbf{mén}to.\

3. Exténdens cælum sicut \textbf{pel}lem:~*  qui tegis aquis superi\textit{ó}\textit{ra} \textbf{e}jus.\

4. Qui ponis nubem ascénsum \textbf{tu}um:~*  qui ámbulas super pen\textit{nas} \textit{ven}\textbf{tó}rum.\

5. Qui facis ángelos tuos, \textbf{spí}ritus:~*  et minístros tuos i\textit{gnem} \textit{u}\textbf{rén}tem.\

6. Qui fundásti terram super stabilitátem \textbf{su}am:~*  non inclinábitur in sǽ\textit{cu}\textit{lum} \textbf{sǽ}culi.\

7. Abýssus, sicut vestiméntum, amíctus \textbf{e}jus:~*  super montes \textit{sta}\textit{bunt} \textbf{a}quæ.\

8. Ab increpatióne tua \textbf{fú}gient:~*  a voce tonítrui tui \textit{for}\textit{mi}\textbf{dá}bunt.\

9. Ascéndunt montes: et descéndunt \textbf{cam}pi~*  in locum, quem fun\textit{dás}\textit{ti} \textbf{e}is.\

10. Términum posuísti, quem non transgredi\textbf{én}tur:~*  neque converténtur ope\textit{rí}\textit{re} \textbf{ter}ram.\

11. Qui emíttis fontes in con\textbf{vál}libus:~*  inter médium móntium pertrans\textit{í}\textit{bunt} \textbf{a}quæ.\

12. Potábunt omnes béstiæ \textbf{a}gri:~*  exspectábunt ónagri in \textit{si}\textit{ti} \textbf{su}a.\

13. Super ea vólucres cæli habi\textbf{tá}bunt:~*  de médio petrárum \textit{da}\textit{bunt} \textbf{vo}ces.\

14. Rigans montes de superióribus \textbf{su}is:~*  de fructu óperum tuórum satiá\textit{bi}\textit{tur} \textbf{ter}ra:\

15. Prodúcens fœnum ju\textbf{mén}tis:~*  et herbam servi\textit{tú}\textit{ti} \textbf{hó}minum:\

16. Ut edúcas panem de \textbf{ter}ra:~*  et vinum lætífi\textit{cet} \textit{cor} \textbf{hó}minis:\

17. Ut exhílaret fáciem in \textbf{ó}leo:~*  et panis cor hómi\textit{nis} \textit{con}\textbf{fír}met.\

18. Saturabúntur ligna campi, et cedri Líbani, quas plan\textbf{tá}vit:~*  illic pásseres ni\textit{di}\textit{fi}\textbf{cá}bunt.\

19. Heródii domus dux est e\textbf{ó}rum:~*  montes excélsi cervis: petra refúgium \textit{he}\textit{ri}\textbf{ná}ciis.\

20. Fecit lunam in \textbf{tém}pora:~*  sol cognóvit oc\textit{cá}\textit{sum} \textbf{su}um.\

21. Posuísti ténebras, et facta \textbf{est} nox:~*  in ipsa pertransíbunt omnes bés\textit{ti}\textit{æ} \textbf{sil}væ.\

22. Cátuli leónum rugiéntes, ut \textbf{rá}piant:~*  et quærant a Deo \textit{es}\textit{cam} \textbf{si}bi.\

23. Ortus est sol, et congre\textbf{gá}ti sunt:~*  et in cubílibus suis col\textit{lo}\textit{ca}\textbf{bún}tur.\

24. Exíbit homo ad opus \textbf{su}um:~*  et ad operatiónem suam us\textit{que} \textit{ad} \textbf{vés}perum.\

25. Quam magnificáta sunt ópera tua, \textbf{Dó}mine!~*  ómnia in sapiéntia fecísti: impléta est terra possessi\textit{ó}\textit{ne} \textbf{tu}a.\

26. Hoc mare magnum, et spatiósum \textbf{má}nibus:~*  illic reptília, quorum \textit{non} \textit{est} \textbf{nú}merus.\

27. Animália pusílla cum \textbf{ma}gnis:~*  illic naves \textit{per}\textit{trans}\textbf{í}bunt.\

28. Draco iste, quem formásti ad illudéndum \textbf{e}i:~*  ómnia a te exspéctant ut des illis es\textit{cam} \textit{in} \textbf{tém}pore.\

29. Dante te illis, \textbf{cól}ligent:~*  aperiénte te manum tuam, ómnia implebúntur \textit{bo}\textit{ni}\textbf{tá}te.\

30. Averténte autem te fáciem, turba\textbf{bún}tur:~*  áuferes spíritum eórum, et defícient, et in púlverem suum \textit{re}\textit{ver}\textbf{tén}tur.\

31. Emíttes spíritum tuum, et crea\textbf{bún}tur:~*  et renovábis fá\textit{ci}\textit{em} \textbf{ter}ræ.\

32. Sit glória Dómini in \textbf{sǽ}culum:~*  lætábitur Dóminus in opé\textit{ri}\textit{bus} \textbf{su}is:\

33. Qui réspicit terram, et facit eam \textbf{tré}mere:~*  qui tangit mon\textit{tes}, \textit{et} \textbf{fú}migant.\

34. Cantábo Dómino in vita \textbf{me}a:~*  psallam Deo meo, \textit{quám}\textit{di}\textbf{u} sum.\

35. Jucúndum sit ei elóquium \textbf{me}um:~*  ego vero delectá\textit{bor} \textit{in} \textbf{Dó}mino.\

36. Defíciant peccatóres a terra, et iníqui ita ut \textbf{non} sint:~*  bénedic, ánima \textit{me}\textit{a}, \textbf{Dó}mino.\

