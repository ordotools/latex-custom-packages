2. Confessiónem, et decórem \textit{ind}\textit{u}\textbf{ís}ti:~*  amíctus lúmine sicut vesti\textbf{mén}to.\

3. Exténdens cælum \textit{sic}\textit{ut} \textbf{pel}lem:~*  qui tegis aquis superióra \textbf{e}jus.\

4. Qui ponis nubem a\textit{scén}\textit{sum} \textbf{tu}um:~*  qui ámbulas super pennas ven\textbf{tó}rum.\

5. Qui facis ángelos \textit{tu}\textit{os}, \textbf{spí}ritus:~*  et minístros tuos ignem u\textbf{rén}tem.\

6. Qui fundásti terram super stabili\textit{tá}\textit{tem} \textbf{su}am:~*  non inclinábitur in sǽculum \textbf{sǽ}culi.\

7. Abýssus, sicut vestiméntum, a\textit{míc}\textit{tus} \textbf{e}jus:~*  super montes stabunt \textbf{a}quæ.\

8. Ab increpatióne \textit{tu}\textit{a} \textbf{fú}gient:~*  a voce tonítrui tui formi\textbf{dá}bunt.\

9. Ascéndunt montes: et de\textit{scén}\textit{dunt} \textbf{cam}pi~*  in locum, quem fundásti \textbf{e}is.\

10. Términum posuísti, quem non trans\textit{gre}\textit{di}\textbf{én}tur:~*  neque converténtur operíre \textbf{ter}ram.\

11. Qui emíttis fontes \textit{in} \textit{con}\textbf{vál}libus:~*  inter médium móntium pertransíbunt \textbf{a}quæ.\

12. Potábunt omnes bés\textit{ti}\textit{æ} \textbf{a}gri:~*  exspectábunt ónagri in siti \textbf{su}a.\

13. Super ea vólucres cæli \textit{ha}\textit{bi}\textbf{tá}bunt:~*  de médio petrárum dabunt \textbf{vo}ces.\

14. Rigans montes de superió\textit{ri}\textit{bus} \textbf{su}is:~*  de fructu óperum tuórum satiábitur \textbf{ter}ra:\

15. Prodúcens fœ\textit{num} \textit{ju}\textbf{mén}tis:~*  et herbam servitúti \textbf{hó}minum:\

16. Ut edúcas pa\textit{nem} \textit{de} \textbf{ter}ra:~*  et vinum lætíficet cor \textbf{hó}minis:\

17. Ut exhílaret fáci\textit{em} \textit{in} \textbf{ó}leo:~*  et panis cor hóminis con\textbf{fír}met.\

18. Saturabúntur ligna campi, et cedri Líbani, \textit{quas} \textit{plan}\textbf{tá}vit:~*  illic pásseres nidifi\textbf{cá}bunt.\

19. Heródii domus dux \textit{est} \textit{e}\textbf{ó}rum:~*  montes excélsi cervis: petra refúgium heri\textbf{ná}ciis.\

20. Fecit lu\textit{nam} \textit{in} \textbf{tém}pora:~*  sol cognóvit occásum \textbf{su}um.\

21. Posuísti ténebras, et \textit{fac}\textit{ta} \textbf{est} nox:~*  in ipsa pertransíbunt omnes béstiæ \textbf{sil}væ.\

22. Cátuli leónum rugién\textit{tes}, \textit{ut} \textbf{rá}piant:~*  et quærant a Deo escam \textbf{si}bi.\

23. Ortus est sol, et \textit{con}\textit{gre}\textbf{gá}ti sunt:~*  et in cubílibus suis colloca\textbf{bún}tur.\

24. Exíbit homo ad \textit{o}\textit{pus} \textbf{su}um:~*  et ad operatiónem suam usque ad \textbf{vés}perum.\

25. Quam magnificáta sunt ópera \textit{tu}\textit{a}, \textbf{Dó}mine!~*  ómnia in sapiéntia fecísti: impléta est terra possessióne \textbf{tu}a.\

26. Hoc mare magnum, et spati\textit{ó}\textit{sum} \textbf{má}nibus:~*  illic reptília, quorum non est \textbf{nú}merus.\

27. Animália pusíl\textit{la} \textit{cum} \textbf{ma}gnis:~*  illic naves pertrans\textbf{í}bunt.\

28. Draco iste, quem formásti ad illu\textit{dén}\textit{dum} \textbf{e}i:~*  ómnia a te exspéctant ut des illis escam in \textbf{tém}pore.\

29. Dante te \textit{il}\textit{lis}, \textbf{cól}ligent:~*  aperiénte te manum tuam, ómnia implebúntur boni\textbf{tá}te.\

30. Averténte autem te fáciem, \textit{tur}\textit{ba}\textbf{bún}tur:~*  áuferes spíritum eórum, et defícient, et in púlverem suum rever\textbf{tén}tur.\

31. Emíttes spíritum tuum, et \textit{cre}\textit{a}\textbf{bún}tur:~*  et renovábis fáciem \textbf{ter}ræ.\

32. Sit glória Dómi\textit{ni} \textit{in} \textbf{sǽ}culum:~*  lætábitur Dóminus in opéribus \textbf{su}is:\

33. Qui réspicit terram, et facit \textit{e}\textit{am} \textbf{tré}mere:~*  qui tangit montes, et \textbf{fú}migant.\

34. Cantábo Dómino in \textit{vi}\textit{ta} \textbf{me}a:~*  psallam Deo meo, quámdi\textbf{u} sum.\

35. Jucúndum sit ei eló\textit{qui}\textit{um} \textbf{me}um:~*  ego vero delectábor in \textbf{Dó}mino.\

36. Defíciant peccatóres a terra, et iníqui i\textit{ta} \textit{ut} \textbf{non} sint:~*  bénedic, ánima mea, \textbf{Dó}mino.\

