\documentclass{article}

% Basic usage - load the psalms package
\usepackage{psalms}

% Set the font (required for LuaLaTeX)
\setmainfont{Latin Modern Roman}

\begin{document}

\section*{Psalms Package - Basic Usage Examples}

% ===== Example 1: Basic psalm with default settings =====
\textbf{Example 1: Psalm 117 with Tone 8G (default positional accent mode)}

\psalm{117}{8G}

\bigskip

% ===== Example 2: Change accent mode inline =====
\textbf{Example 2: Psalm 117 with Tone 1 (switching to orthographic mode)}

\SetPsalmAccentMode{orthographic}
\psalm{117}{1}

\bigskip

% ===== Example 3: Back to positional mode =====
\textbf{Example 3: Psalm 150 with Tone 2D (positional mode)}

\SetPsalmAccentMode{positional}
\psalm{150}{2D}

\bigskip

% ===== Example 4: Different tones =====
\textbf{Example 4: Psalm 113 with Tone 3}

\psalm{113}{3}

\bigskip

\textbf{Example 5: Psalm 116 with Tone 4}

\psalm{116}{4}

\bigskip

% ===== Example 6: Single line test =====
\textbf{Example 6: Testing individual lines with \textbackslash PsalmLine}

\PsalmLine{Beatus vir qui non abiit in consilio impiorum * et in via peccatorum non stetit.}\par

\bigskip

\PsalmLine{Dominum laudate omnes gentes * laudate eum omnes populi.}\par

\bigskip

% ===== Example 7: Legacy command format =====
\textbf{Example 7: Using legacy command format (backward compatibility)}

\psalmlegacy[8G]{117}

\end{document}

