2. Quia ipse super mária fun\textbf{dá}vit \textbf{e}um:~*  et super flúmina præpa\textbf{rá}vit \textbf{e}um.\

3. Quis ascéndet in \textbf{mon}tem \textbf{Dó}\textbf{mi}ni?~*  aut quis stabit in loco \textbf{sanc}to \textbf{e}jus?\

4. Innocens mánibus et mundo corde, \dag\  qui non accépit in vano \textbf{á}nimam \textbf{su}am,~*  nec jurávit in dolo \textbf{pró}ximo \textbf{su}o.\

5. Hic accípiet benedicti\textbf{ó}nem a \textbf{Dó}\textbf{mi}no:~*  et misericórdiam a Deo, salu\textbf{tá}ri \textbf{su}o.\

6. Hæc est generátio quæ\textbf{rén}tium \textbf{e}um,~*  quæréntium fáciem \textbf{De}i \textbf{Ja}cob.\

7. Attóllite portas, príncipes, vestras, \dag\  et elevámini, portæ \textbf{æ}ter\textbf{ná}les:~*  et intro\textbf{í}bit Rex \textbf{gló}riæ.\

8. Quis est iste Rex glóriæ? \dag\  Dóminus \textbf{for}tis et \textbf{pot}ens:~*  Dóminus \textbf{pot}ens in \textbf{prǽ}lio.\

9. Attóllite portas, príncipes, vestras, \dag\  et elevámini, portæ \textbf{æ}ter\textbf{ná}les:~*  et intro\textbf{í}bit Rex \textbf{gló}riæ.\

10. Quis est \textbf{is}te Rex \textbf{gló}\textbf{ri}æ?~*  Dóminus virtútum ipse \textbf{est} Rex \textbf{gló}riæ.\

11. Glória \textbf{Pa}tri, et \textbf{Fí}\textbf{li}o,~*  et Spi\textbf{rí}tui \textbf{Sanc}to.\

12. Sicut erat in princípio, et \textbf{nunc}, et \textbf{sem}per,~*  et in sǽcula sæcu\textbf{ló}rum. \textbf{A}men.\

